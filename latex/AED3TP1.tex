\documentclass[a4paper]{article}

\setlength{\parskip}{0.1em}
\input{Algo1Macros}
\usepackage{caratula}

\begin{document}

\titulo{Informe de Trabajo Pr\'actico 1}
\subtitulo{Subset Sum Problem}
\fecha{...}
\materia{Algoritmos y Estructuras de Datos III}
\grupo{}
\newcommand{\senial}{\textit{se\~nal}}
\newcommand{\amplitud}{\textit{amplitud}}
\newcommand{\tiempo}{\textit{tiempo}}
\newcommand{\intervalo}{\textit{intervalo}}

% Pongan cuantos integrantes quieran
\integrante{Springhart, Gonzalo}{308/17}{glspringhart@gmail.com}

\maketitle

\section*{Introducci\'on}

En este informe vamos a comparar a eficiencia de distintos algoritmos utilizados para resolver un problema conocido como \textit{Subset Sum Problem} (o Problema de suma de subconjuntos). El mismo consiste en lo siguiente, dado un conjunto $S$ de $n$ elementos, cada uno con un valor asociado $v_i$ y un valor objetivo $V$, se quiere saber si existe un subconjunto de \'items de $S$ que sumen exactamente el valor objetivo, y si existe dicho subconjunto, se quiere saber cu\'al es la m\'inima cardinalidad entre todos los subconjuntos posibles, en otras palabras, hay que decidir si existe $R \subseteq S$ tal que $\sum_{i \in R} vi = V$. Se asumen tambi\'en que los valores de $S$ son enteros no negativos (aunque el problema se puede resolver también sin necesidad de esta restricci\'on).
%INSERTAR ACA EJEMPLOS DEL PROBLEMAAAAAAAAAAAA
\\
El objetivo es ver cu\'al de los algoritmos es m\'as eficiente al resolver el problema, se van a presentar 4 algoritmos que resuelven el problema, indicando como funcionan, justificando sus complejidades y comprobando a trav\'ez de experimentos que estas complejidades son ciertas.

\section{Algoritmos}

\subsection{Fuerza Bruta}


\end{document}
