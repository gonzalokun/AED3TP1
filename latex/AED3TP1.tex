\documentclass[a4paper]{article}

\setlength{\parskip}{0.1em}
\input{Algo1Macros}
\usepackage{caratula}
\usepackage{algorithm}
\usepackage{algpseudocode}
%\usepackage{algorithmicx}

\begin{document}

\titulo{Informe de Trabajo Pr\'actico 1}
\subtitulo{Subset Sum Problem}
\fecha{...}
\materia{Algoritmos y Estructuras de Datos III}
\grupo{}
\newcommand{\senial}{\textit{se\~nal}}
\newcommand{\amplitud}{\textit{amplitud}}
\newcommand{\tiempo}{\textit{tiempo}}
\newcommand{\intervalo}{\textit{intervalo}}

% Pongan cuantos integrantes quieran
\integrante{Springhart, Gonzalo}{308/17}{glspringhart@gmail.com}

\maketitle

\section*{Introducci\'on}

En este informe vamos a comparar a eficiencia de distintos algoritmos utilizados para resolver un problema conocido como \textit{Subset Sum Problem} (o Problema de suma de subconjuntos). El mismo consiste en lo siguiente, dado un conjunto $S$ de $n$ elementos, cada uno con un valor asociado $v_i$ y un valor objetivo $V$, se quiere saber si existe un subconjunto de \'items de $S$ que sumen exactamente el valor objetivo, y si existe dicho subconjunto, se quiere saber cu\'al es la m\'inima cardinalidad entre todos los subconjuntos posibles, en otras palabras, hay que decidir si existe $R \subseteq S$ tal que $\sum_{i \in R} vi = V$. Se asumen tambi\'en que los valores de $S$ son enteros no negativos (aunque el problema se puede resolver también sin necesidad de esta restricci\'on).
%INSERTAR ACA EJEMPLOS DEL PROBLEMAAAAAAAAAAAA
\\
El objetivo es ver cu\'al de los algoritmos es m\'as eficiente al resolver el problema, se van a presentar 4 algoritmos que resuelven el problema, indicando como funcionan, justificando sus complejidades y comprobando a trav\'ez de experimentos que estas complejidades son ciertas.

\section{Algoritmos y justificaci\'on de complejidades}

Se va a usar la siguiente notaci\'on:
\begin{itemize}
	\item $S$ es el conjunto, que tiene $n$ elementos, cada uno con un valor asociado $v_i$ con $i \in \{1,...,n\}$
	\item $V$ es el valor objetivo
\end{itemize}

\subsection{Fuerza Bruta}
El primer algoritmo presentado para resolver el problema es uno de \textit{Fuerza Bruta}, b\'asicamente el algoritmo genera todos los conjuntos posibles con los elementos de $S$ y se fija cu\'al de ellos tiene elementos tales que su suma de exactamente $V$, mientras los calcula se va quedando con el que tiene menor cardinalidad.
\\
El algoritmo es el siguiente:
\\
%ACA VA EL ALGO
\begin{algorithm}
\begin{algorithmic}
	\Procedure{fuerzaBruta}{S, V}
		\State $longMinima \gets -1$
		\State $partes \gets generarConjPartes(S)$
		\ForAll{Conjunto $c \in partes$}
			\State $sum \gets sumarElem(c)$
			\If{$sum == V$}
				\If{$longMinima == -1$}
					\State $longMinima \gets |c|$
				\Else
					\State $longMinima \gets min(longMinima, |c|)$
				\EndIf
			\EndIf
		\EndFor
		
		\Return{$longMinima$}
	\EndProcedure
\end{algorithmic}
\end{algorithm}

%EXPLICAR LAS FUNCIONES DE ADENTRO DEL ALGOOOOOOO
Como se puede ver el algoritmo ejecuta un ciclo $2^n$ veces, dentro de cada ciclo se realiza la suma de sus elementos, esta suma se puede realizar en tiempo lineal, y las operaciones para calcular el m\'inimo despu\'es de hacer la suma son $O(1)$
Entonces la complejidad del algoritmo es $O(n * 2^n)$.

\subsection{Backtracking}
El Backtracking es un algoritmo que se utiliza para encontrar todas o algunas de las soluciones de alg\'un problema, se basa en ir armando la soluci\'on correcta al problema desechando las que no pueden ser correctas a medida que se ejecuta.
\subsection{Programaci\'on Din\'amica Top Down}
\subsection{Programaci\'on Din\'amica Bottom Up}

\end{document}
